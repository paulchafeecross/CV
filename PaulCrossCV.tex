\documentclass[12pt,]{article}
\usepackage[sc, osf]{mathpazo}
\usepackage{amssymb,amsmath}
\usepackage{ifxetex,ifluatex}
\usepackage{fixltx2e} % provides \textsubscript
\ifnum 0\ifxetex 1\fi\ifluatex 1\fi=0 % if pdftex
  \usepackage[T1]{fontenc}
  \usepackage[utf8]{inputenc}
\else % if luatex or xelatex
  \ifxetex
    \usepackage{mathspec}
  \else
    \usepackage{fontspec}
  \fi
  \defaultfontfeatures{Ligatures=TeX,Scale=MatchLowercase}
\fi
% use upquote if available, for straight quotes in verbatim environments
\IfFileExists{upquote.sty}{\usepackage{upquote}}{}
% use microtype if available
\IfFileExists{microtype.sty}{%
\usepackage{microtype}
\UseMicrotypeSet[protrusion]{basicmath} % disable protrusion for tt fonts
}{}
\usepackage[margin=1in]{geometry}
\usepackage{hyperref}
\PassOptionsToPackage{usenames,dvipsnames}{color} % color is loaded by hyperref
\hypersetup{unicode=true,
            pdftitle={Paul Chafee Cross},
            colorlinks=true,
            linkcolor=blue,
            citecolor=Blue,
            urlcolor=blue,
            breaklinks=true}
\urlstyle{same}  % don't use monospace font for urls
\usepackage{graphicx,grffile}
\makeatletter
\def\maxwidth{\ifdim\Gin@nat@width>\linewidth\linewidth\else\Gin@nat@width\fi}
\def\maxheight{\ifdim\Gin@nat@height>\textheight\textheight\else\Gin@nat@height\fi}
\makeatother
% Scale images if necessary, so that they will not overflow the page
% margins by default, and it is still possible to overwrite the defaults
% using explicit options in \includegraphics[width, height, ...]{}
\setkeys{Gin}{width=\maxwidth,height=\maxheight,keepaspectratio}
\IfFileExists{parskip.sty}{%
\usepackage{parskip}
}{% else
\setlength{\parindent}{0pt}
\setlength{\parskip}{6pt plus 2pt minus 1pt}
}
\setlength{\emergencystretch}{3em}  % prevent overfull lines
\providecommand{\tightlist}{%
  \setlength{\itemsep}{0pt}\setlength{\parskip}{0pt}}
\setcounter{secnumdepth}{0}
% Redefines (sub)paragraphs to behave more like sections
\ifx\paragraph\undefined\else
\let\oldparagraph\paragraph
\renewcommand{\paragraph}[1]{\oldparagraph{#1}\mbox{}}
\fi
\ifx\subparagraph\undefined\else
\let\oldsubparagraph\subparagraph
\renewcommand{\subparagraph}[1]{\oldsubparagraph{#1}\mbox{}}
\fi

%%% Use protect on footnotes to avoid problems with footnotes in titles
\let\rmarkdownfootnote\footnote%
\def\footnote{\protect\rmarkdownfootnote}

%%% Change title format to be more compact
\usepackage{titling}

% Create subtitle command for use in maketitle
\providecommand{\subtitle}[1]{
  \posttitle{
    \begin{center}\large#1\end{center}
    }
}

\setlength{\droptitle}{-2em}

  \title{Paul Chafee Cross}
    \pretitle{\vspace{\droptitle}\centering\huge}
  \posttitle{\par}
    \author{}
    \preauthor{}\postauthor{}
    \date{}
    \predate{}\postdate{}
  

\begin{document}
\maketitle

\hrule
\centering

Wildlife Researcher, Terrestrial Ecology Branch Chief,\\
Northern Rocky Mountain Science Center, U.S. Geological Survey\\
2327 University Way, Suite 2, Bozeman, MT 59715\\
\href{http://www.usgs.gov/staff-profiles/paul-cross}{www.usgs.gov/staff-profiles/paul-cross},\\
\href{http://orcid.org/0000-0001-8045-5213}{orcid:
0000-0001-8045-5213}\\
email: \href{mailto:pcross@usgs.gov}{\nolinkurl{pcross@usgs.gov}}\\
cellphone: 406-581-1763\\
Updated: September 12, 2019

\hrule

\raggedright

\hypertarget{education}{%
\section{Education}\label{education}}

Ph.D.~Environmental Science, Policy \& Mang., University of California,
Berkeley, CA \hfill 2005\\
B.A., Environmental Science, University of Virginia, Charlottesville, VA
\hfill 1998

\hypertarget{research-experience}{%
\section{Research Experience}\label{research-experience}}

Wildlife Health Researcher, U.S. Geological Survey \hfill 2005-present\\
Faculty Affiliate, Montana State University \hfill 2006-present\\
Graduate Student Researcher, U.C. Berkeley \hfill 1999-2005

\hypertarget{publications}{%
\section{Publications}\label{publications}}

\hypertarget{in-review}{%
\subsection{In review}\label{in-review}}

Brandell, E, E. S. Almberg, P. C. Cross, A. Dobson, D. Smith, and P.
Hudson (2020). ``The invasion, dynamics, and consequences of infectious
diseases in Yellowstone's wolves''. In: \emph{Yellowstone Wolves: Two
Decades of Science and Discovery}. Ed. by D. Smith, D. MacNulty and D.
R. Stahler. Chicago: University of Chicago.

Cotterill, G. G, P. C. Cross, J. A. Merkle, J. D. Rogerson, B. M.
Scurlock, and J. T. Du Toit (2020). ``Parsing the effects of management
efforts and environmental drivers on a chronic wildlife disease''. In:
\emph{Journal of Applied Ecology} In Review, pp.~000-000.

Kamath, P. L, K. Manlove, E. F. Cassirer, P. C. Cross, and T. E. Besser
(2020). ``Genetic structure of Mycoplasma ovipneumoniae informs pathogen
spillover dynamics between domestic and bighorn sheep in the western
United States''. In: \emph{Scientific Reports} In review, pp.~000-000.

Rayl, N. D, J. A. Merkle, K. M. Proffitt, E. S. Almberg, J. D. Jones, J.
A. Gude, and P. C. Cross (2020). ``Elk migration influences the risk of
disease spillover in the Greater Yellowstone Ecosystem''. In:
\emph{Journal of Animal Ecology} In review, pp.~000-000.

\hypertarget{section}{%
\subsection{2019}\label{section}}

Cross, P, D. J. Prosser, A. M. Ramey, E. M. Hanks, and K. M. Pepin
(2019). ``Confronting models with data: The challenges of estimating
disease spillover''. In: \emph{Philosophical Transactions of the Royal
Society B: Biological Sciences} 374. DOI:
\href{https://doi.org/10.1098\%2Frstb.2018.0435}{10.1098/rstb.2018.0435}.
URL: \url{https://doi.org/10.1098/rstb.2018.0435}.

Manlove, K, M. Branan, R. S. Miller, E. F. Cassirer, K. Baker, D.
Bradway, S. Sweeney, K. L. Marshall, P. C. Cross, and T. E. Besser
(2019). ``Risk factors and productivity losses associated with
Mycoplasma ovipneumoniae infection in United States domestic sheep
operations''. In: \emph{Preventative Veterinary Medicine} 168,
pp.~30-38. DOI:
\href{https://doi.org/10.1016\%2Fj.prevetmed.2019.04.006}{10.1016/j.prevetmed.2019.04.006}.
URL:
\url{https://www.sciencedirect.com/science/article/pii/S0167587718304446}.

Manlove, K, L. M. Sampson, B. Borremans, E. Cassirer, R. Miller, K.
Pepin, T. Besser, and P. Cross (2019). ``Epidemic growth rates and host
movement patterns shape management performance for pathogen spillover at
the wildlife--livestock interface''. In: \emph{Philosophical
Transactions of the Royal Society B: Biological Sciences} 374. DOI:
\href{https://doi.org/10.1098\%2Frstb.2018.0343}{10.1098/rstb.2018.0343}.
URL: \url{https://doi.org/10.1098/rstb.2018.0343}.

Rayl, N. D, K. M. Proffitt, E. S. Almberg, J. A. Merkle, J. H. Jones, J.
A. Gude, and P. C. Cross (2019). ``Modelling Elk-to-Livestock
Transmission Risk to Predict Hotspots of Brucellosis Spillover''. In:
\emph{Journal of Wildlife Management} 83.4, pp.~817-829. DOI:
\href{https://doi.org/10.1002\%2Fjwmg.21645}{10.1002/jwmg.21645}. URL:
\url{https://doi.org/10.1002/jwmg.21645}.

Sokolow, S. H, N. Nova, K. Pepin, A. J. Peel, J. R. Pulliam, K. Manlove,
P. C. Cross, D. J. Becker, R. K. Plowright, H. McCallum, and G. A. De
Leo (2019). ``Ecological interventions to prevent and manage zoonotic
pathogen spillover''. In: \emph{Philosophical Transactions of the Royal
Society B: Biological Sciences} 374. DOI:
\href{https://doi.org/10.1098\%2Frstb.2018.0342}{10.1098/rstb.2018.0342}.
URL: \url{https://doi.org/10.1098/rstb.2018.0342}.

Wijeyakulasuriya, D, E. Hank, B. Shaby, and P. Cross (2019). ``Extreme
value based methods for modeling elk yearly movements''. In:
\emph{Journal of Agricultural, Biological and Environmental Statistics}
24.1, pp.~73-91. DOI:
\href{https://doi.org/10.1007\%2Fs13253-018-00342-2}{10.1007/s13253-018-00342-2}.

\hypertarget{section-1}{%
\subsection{2018}\label{section-1}}

Astorga, F, S. Carver, E. S. Almberg, G. R. Sousa, K. Wingfield, K. D.
Niedringhaus, P. Van Wick, L. Rossi, Y. Xie, P. Cross, S. Angelone, C.
Gortázar, and L. E. Escobar (2018). ``International meeting on sarcoptic
mange in wildlife, June 2018, Blacksburg, Virginia, USA''. In:
\emph{Parasites \& Vectors} 11, p.~449. ISSN: 1756-3305. DOI:
\href{https://doi.org/10.1186\%2Fs13071-018-3015-1}{10.1186/s13071-018-3015-1}.

Brennan, A, E. M. Hanks, J. A. Merkle, E. K. Cole, S. R. Dewey, A. B.
Courtemanch, and P. C. Cross (2018). ``Examining speed versus selection
in connectivity models using elk migration as an example''. In:
\emph{Landscape Ecology} 33.6, pp.~955-968. ISSN: 0921-2973 1572-9761.
DOI:
\href{https://doi.org/10.1007\%2Fs10980-018-0642-z}{10.1007/s10980-018-0642-z}.

Cotterill, G. G, P. C. Cross, E. K. Cole, R. K. Fuda, J. D. Rogerson, B.
M. Scurlock, and J. T. du Toit (2018). ``Winter feeding of elk in the
Greater Yellowstone Ecosystem and its effects on disease dynamics''. In:
\emph{Philos Trans R Soc Lond B Biol Sci} 373.1745, p.~20170093. ISSN:
1471-2970 (Electronic) 0962-8436 (Linking). DOI:
\href{https://doi.org/10.1098\%2Frstb.2017.0093}{10.1098/rstb.2017.0093}.
URL: \url{https://www.ncbi.nlm.nih.gov/pubmed/29531148}.

Cotterill, G. G, P. C. Cross, A. D. Middleton, J. D. Rogerson, B. M.
Scurlock, and J. T. du Toit (2018). ``Hidden cost of disease in a
free-ranging ungulate: brucellosis reduces mid-winter pregnancy in
elk''. In: \emph{Ecol Evol} 8.22, pp.~10733-10742. ISSN: 2045-7758
(Print) 2045-7758 (Linking). DOI:
\href{https://doi.org/10.1002\%2Fece3.4521}{10.1002/ece3.4521}. URL:
\url{https://www.ncbi.nlm.nih.gov/pubmed/30519402}.

Cross, P. C, F. T. van Manen, M. Viana, E. S. Almberg, D. Bachen, E. E.
Brandell, M. A. Haroldson, P. J. Hudson, D. R. Stahler, and D. W. Smith
(2018). ``Estimating distemper virus dynamics among wolves and grizzly
bears using serology and Bayesian state-space models''. In: \emph{Ecol
Evol} 8.17, pp.~8726-8735. ISSN: 2045-7758 (Print) 2045-7758 (Linking).
DOI: \href{https://doi.org/10.1002\%2Fece3.4396}{10.1002/ece3.4396}.
URL: \url{https://www.ncbi.nlm.nih.gov/pubmed/30271540}.

Haggerty, J. H, K. Epstein, M. Stone, and P. C. Cross (2018). ``Land Use
Diversification and Intensification on Elk Winter Range in Greater
Yellowstone: Framework and Agenda for Social-Ecological Research''. In:
\emph{Rangeland Ecology \& Management} 71.2, pp.~171-174. ISSN:
15507424. DOI:
\href{https://doi.org/10.1016\%2Fj.rama.2017.11.002}{10.1016/j.rama.2017.11.002}.

Huyvaert, K. P, R. E. Russell, K. A. Patyk, M. E. Craft, P. C. Cross, M.
G. Garner, M. K. Martin, P. Nol, and D. P. Walsh (2018). ``Challenges
and Opportunities Developing Mathematical Models of Shared Pathogens of
Domestic and Wild Animals''. In: \emph{Vet Sci} 5.4. ISSN: 2306-7381
(Electronic) 2306-7381 (Linking). DOI:
\href{https://doi.org/10.3390\%2Fvetsci5040092}{10.3390/vetsci5040092}.
URL: \url{https://www.ncbi.nlm.nih.gov/pubmed/30380736}.

Manlove, K, C. M. Aiello, P. Sah, B. Cummins, P. J. Hudson, and P. C.
Cross (2018). ``The ecology of movement and behavior: A saturated
tripartite model for describing animal contacts''. In: \emph{Proceedings
of the Royal Society B: Biological Sciences} 285, p.~20180670. DOI:
\href{https://doi.org/10.1098\%2Frspb.2018.0670}{10.1098/rspb.2018.0670}.

Merkle, J. A, P. C. Cross, B. M. Scurlock, E. K. Cole, A. B.
Courtemanch, S. R. Dewey, and M. J. Kauffman (2018). ``Linking spring
phenology with mechanistic models of host movement to predict disease
transmission risk''. In: \emph{Journal of Applied Ecology} 55,
pp.~810-819. ISSN: 00218901. DOI:
\href{https://doi.org/10.1111\%2F1365-2664.13022}{10.1111/1365-2664.13022}.

Rayl, N. D., K. M. Proffitt, E. S. Almberg, J. A. Merkle, J. H. Jones,
J. A. Gude, and P. C. Cross (2018). \emph{Modelling elk-to-livestock
transmission risk to identify hotspots of brucellosis spillover}.
Report. Montana Fish, Wildlife and Parks.

\hypertarget{section-2}{%
\subsection{2017}\label{section-2}}

Benavides, J. A, D. Caillaud, B. M. Scurlock, E. J. Maichak, W. H.
Edwards, and P. C. Cross (2017). ``Estimating loss of Brucella abortus
antibodies from age-specific serological data in elk''. In:
\emph{EcoHealth} 14.2, pp.~234-243. ISSN: 1612-9210 (Electronic)
1612-9202 (Linking). DOI:
\href{https://doi.org/10.1007\%2Fs10393-017-1235-z}{10.1007/s10393-017-1235-z}.
URL: \url{https://www.ncbi.nlm.nih.gov/pubmed/28508154}.

Brennan, A, P. C. Cross, K. Portacci, B. M. Scurlock, and W. H. Edwards
(2017). ``Shifting brucellosis risk in livestock coincides with
spreading seroprevalence in elk''. In: \emph{PLoS One} 12.6,
p.~e0178780. DOI:
\href{https://doi.org/10.1371\%2Fjournal.pone.0178780}{10.1371/journal.pone.0178780}.
URL: \url{https://www.ncbi.nlm.nih.gov/pubmed/28609437}.

Cassirer, E. F, K. R. Manlove, E. S. Almberg, P. L. Kamath, M. Cox, P.
Wolff, A. Roug, J. Shannon, R. Robinson, R. B. Harris, B. J. Gonzales,
R. K. Plowright, P. J. Hudson, P. C. Cross, A. Dobson, et al. (2017).
``Pneumonia in bighorn sheep: Risk and resilience''. In: \emph{The
Journal of Wildlife Management}. ISSN: 0022541X. DOI:
\href{https://doi.org/10.1002\%2Fjwmg.21309}{10.1002/jwmg.21309}.

Maichak, E. J, B. M. Scurlock, P. C. Cross, J. D. Rogerson, W. H.
Edwards, B. Wise, S. G. Smith, and T. J. Kreeger (2017). ``Assessment of
the Brucella abortus Strain 19 ballistic vaccination program in elk on
winter feedgrounds of Wyoming, USA''. In: \emph{Wildlife Society
Bulletin} 41.1, pp.~70-79. DOI:
\href{https://doi.org/10.1002\%2Fwsb.734}{10.1002/wsb.734}.

Manlove, K. R, E. F. Cassirer, R. K. Plowright, P. C. Cross, and P. J.
Hudson (2017). ``Contact and contagion: Bighorn sheep demographic states
vary in probability of transmission given contact''. In: \emph{Journal
of Animal Ecology} 86, pp.~908-920. DOI:
\href{https://doi.org/10.1111\%2F1365-2656.12664}{10.1111/1365-2656.12664}.

National Academies of Sciences, E. and Medicine (2017). \emph{Revisiting
Brucellosis in the Greater Yellowstone Area}. Washington, DC: The
National Academies Press, p.~209. ISBN: 978-0-309-45831-3. DOI:
\href{https://doi.org/10.17226\%2F24750}{10.17226/24750}. URL:
\url{https://doi.org/10.17226/24750}.

Pepin, K. M, S. L. Kay, B. D. Golas, S. S. Shriner, A. T. Gilbert, R. S.
Miller, A. L. Graham, S. Riley, P. C. Cross, M. D. Samuel, M. B. Hooten,
J. A. Hoeting, J. O. Lloyd-Smith, C. T. Webb, and M. G. Buhnerkempe
(2017). ``Inferring infection hazard in wildlife populations by linking
data across individual and population scales''. In: \emph{Ecology
Letters} 20.3, pp.~275-292. DOI:
\href{https://doi.org/10.1111\%2Fele.12732}{10.1111/ele.12732}.

Sah, P, S. Leu, P. C. Cross, P. J. Hudson, and S. Bansal (2017).
``Unraveling the disease consequences and mechanisms of modular
structure in animal social networks''. In: \emph{Proceedings of the
National Academy of Science of the United States of America} 114.16,
pp.~4165-4170. DOI:
\href{https://doi.org/10.1073\%2Fpnas.1613616114}{10.1073/pnas.1613616114}.

Toit, J. du, P. Cross, and M. Valeix (2017). ``Weaving wildlife into a
framework for rangeland resilience''. In: \emph{Rangeland Systems:
Processes, Management and Challenges}. Ed. by D. Briske. Springer,
pp.~395-428. DOI:
\href{https://doi.org/10.1007\%2F978-3-319-46709-2}{10.1007/978-3-319-46709-2}.

\hypertarget{section-3}{%
\subsection{2016}\label{section-3}}

Almberg, E. S, P. C. Cross, P. J. Hudson, A. P. Dobson, D. W. Smith, and
D. R. Stahler (2016). ``Infectious diseases of wolves in Yellowstone''.
In: \emph{Yellowstone Science} 24.1, pp.~47-49. URL:
\url{www.nps.gov/yell/learn/ys-24-1-infectious-diseases-of-wolves-in-yellowstone.htm}.

Cross, P. C, E. S. Almberg, C. G. Haase, P. J. Hudson, S. K. Maloney, M.
C. Metz, A. J. Munn, P. Nugent, O. Putzeys, D. R. Stahler, A. C.
Stewart, and D. W. Smith (2016). ``Energetic costs of mange in
Yellowstone wolves estimated from infrared thermography''. In:
\emph{Ecology} 97.8, pp.~1938-1948. DOI:
\href{https://doi.org/10.1890\%2F15-1346.1}{10.1890/15-1346.1}.

Ebinger, M. R, M. A. Haroldson, F. T. van Manen, C. M. Costello, D. D.
Bjornlie, D. J. Thompson, K. A. Gunther, J. K. Fortin, J. E. Teisberg,
S. R. Pils, P. J. White, S. L. Cain, and P. C. Cross (2016). ``Detecting
grizzly bear use of ungulate carcasses using global positioning system
telemetry and activity data''. In: \emph{Oecologia} 181.3, pp.~695-708.
ISSN: 1432-1939 (Electronic) 0029-8549 (Linking). DOI:
\href{https://doi.org/10.1007\%2Fs00442-016-3594-5}{10.1007/s00442-016-3594-5}.

Kamath, P, J. Foster, K. Drees, C. Quance, G. Luikart, N. Anderson, P.
Clarke, E. Cole, W. Edwards, J. Rhyan, J. Treanor, R. Wallen, S.
Robbe-Austerman, and P. Cross (2016). ``Genomics reveals historic and
contemporary transmission dynamics of a bacterial disease among wildlife
and livestock''. In: \emph{Nature Communications} 7, p.~11448. DOI:
\href{https://doi.org/10.1038\%2Fncomms11448}{10.1038/ncomms11448}.

Leach, C, C. Webb, and P. Cross (2016). ``When environmentally
persistent pathogens transform good habitat into ecological traps''. In:
\emph{Royal Society Open Science} 3, p.~160051. DOI:
\href{https://doi.org/10.1098\%2Frsos.160051}{10.1098/rsos.160051}.

Manlove, K. R, J. G. Walker, M. E. Craft, K. P. Huyvaert, M. B. Joseph,
R. S. Miller, P. Nol, K. A. Patyk, D. O'Brien, D. P. Walsh, and P. C.
Cross (2016). "``One Health'' or three? Publication silos among the one
health disciplines". In: \emph{PLoS Biology} 14.4, p.~e1002448. DOI:
\href{https://doi.org/10.1371\%2Fjournal.pbio.1002448}{10.1371/journal.pbio.1002448}.

Manlove, K, E. F. Cassirer, P. C. Cross, R. K. Plowright, and P. J.
Hudson (2016). ``Disease introduction is associated with a phase
transition in bighorn sheep demographics''. In: \emph{Ecology} 97.10,
pp.~2593-2602. DOI:
\href{https://doi.org/10.1002\%2Fecy.1520}{10.1002/ecy.1520}.

\hypertarget{section-4}{%
\subsection{2015}\label{section-4}}

Almberg, E. S, P. C. Cross, A. P. Dobson, D. W. Smith, M. C. Metz, D. R.
Stahler, and P. J. Hudson (2015). ``Social living mitigates the costs of
a chronic illness in a cooperative carnivore''. In: \emph{Ecology
Letters} 18.7, pp.~660-7. DOI:
\href{https://doi.org/10.1111\%2Fele.12444}{10.1111/ele.12444}.

Brennan, A, P. C. Cross, and S. Creel (2015). ``Managing more than the
mean: using quantile regression to identify factors related to large elk
groups''. In: \emph{Journal of Applied Ecology} 52, pp.~1656-1664. ISSN:
00218901. DOI:
\href{https://doi.org/10.1111\%2F1365-2664.12514}{10.1111/1365-2664.12514}.

Cole, E. K, A. M. Foley, J. M. Warren, B. L. Smith, S. R. Dewey, D. G.
Brimeyer, W. S. Fairbanks, H. Sawyer, and P. C. Cross (2015). ``Changing
migratory patterns in the Jackson elk herd''. In: \emph{Journal of
Wildlife Management} 79.6, pp.~877-886. DOI:
\href{https://doi.org/10.1002\%2Fjwmg.917}{10.1002/jwmg.917}.

Cross, P. C, E. J. Maichak, J. D. Rogerson, K. M. Irvine, J. D. Jones,
D. M. Heisey, W. H. Edwards, and B. M. Scurlock (2015). ``Estimating the
phenology of elk brucellosis transmission with hierarchical models of
cause-specific and baseline hazards''. In: \emph{Journal of Wildlife
Management} 79.5, pp.~739-748. DOI:
\href{https://doi.org/10.1002\%2Fjwmg.883}{10.1002/jwmg.883}.

Foley, A. M, P. C. Cross, D. A. Christianson, B. M. Scurlock, and S.
Creel (2015). ``Influences of supplemental feeding on winter elk
calf:cow ratios in the southern Greater Yellowstone Ecosystem''. In:
\emph{Journal of Wildlife Management} 79.6, pp.~887-897. DOI:
\href{https://doi.org/10.1002\%2Fjwmg.908}{10.1002/jwmg.908}.

Gorsich, E. E, V. O. Ezenwa, P. C. Cross, R. G. Bengis, and A. E. Jolles
(2015). ``Context-dependent survival, fecundity and predicted
population-level consequences of brucellosis in African buffalo''. In:
\emph{Journal of Animal Ecology} 84.4, pp.~999-1009. DOI:
\href{https://doi.org/10.1111\%2F1365-2656.12356}{10.1111/1365-2656.12356}.

Sepulveda, M, K. Pelican, P. Cross, A. Eguren, and R. Singer (2015).
``Fine-scale movements of rural free-ranging dogs in conservation areas
in the temperate rainforest of the coastal range of southern Chile''.
In: \emph{Mammalian Biology} 80.4, pp.~290-297. ISSN: 16165047. DOI:
\href{https://doi.org/10.1016\%2Fj.mambio.2015.03.001}{10.1016/j.mambio.2015.03.001}.

\hypertarget{section-5}{%
\subsection{2014}\label{section-5}}

Benavides, J. A, P. C. Cross, G. Luikart, and S. Creel (2014).
``Limitations to estimating bacterial cross-species transmission using
genetic and genomic markers: inferences from simulation modeling''. In:
\emph{Evolutionary Applications} 7.7, pp.~774-87. DOI:
\href{https://doi.org/10.1111\%2Feva.12173}{10.1111/eva.12173}.

Brennan, A, P. C. Cross, M. D. Higgs, W. H. Edwards, B. M. Scurlock, and
S. Creel (2014). ``A multi-scale assessment of animal aggregation
patterns to understand increasing pathogen seroprevalence''. In:
\emph{Ecosphere} 5.10, p.~art138. ISSN: 2150-8925. DOI:
\href{https://doi.org/10.1890\%2Fes14-00181.1}{10.1890/es14-00181.1}.

Hand, B. K, S. Chen, N. Anderson, A. Beja-Pereira, P. C. Cross, M.
Ebinger, H. Edwards, R. A. Garrott, M. D. Kardos, M. Kauffman, E. L.
Landguth, A. Middleton, B. Scurlock, P. J. White, P. Zager, et al.
(2014). ``Sex-biased gene flow among elk in the Greater Yellowstone
Ecosystem''. In: \emph{Journal of Fish and Wildlife Management} 5.1,
pp.~124-132. ISSN: 1944-687X. DOI:
\href{https://doi.org/10.3996\%2F022012-jfwm-017}{10.3996/022012-jfwm-017}.

Jones, J. D, M. J. Kauffman, K. L. Monteith, B. M. Scurlock, S. E.
Albeke, and P. C. Cross (2014). ``Supplemental feeding alters migration
of a temperate ungulate''. In: \emph{Ecological Applications} 24.7,
pp.~1769-1779. DOI:
\href{https://doi.org/10.1890\%2F13-2092.1}{10.1890/13-2092.1}.

Kamath, P. L, D. Elleder, L. Bao, P. C. Cross, J. H. Powell, and M. Poss
(2014). ``The population history of endogenous retroviruses in mule deer
(Odocoileus hemionus)''. In: \emph{Journal of Heredity} 105.2,
pp.~173-87. DOI:
\href{https://doi.org/10.1093\%2Fjhered\%2Fest088}{10.1093/jhered/est088}.

Manlove, K. R, E. F. Cassirer, P. C. Cross, R. K. Plowright, and P. J.
Hudson (2014). ``Costs and benefits of group living with disease: a case
study of pneumonia in bighorn lambs (Ovis canadensis)''. In: \emph{Proc
Biol Sci} 281.1797, p.~20142331. DOI:
\href{https://doi.org/10.1098\%2Frspb.2014.2331}{10.1098/rspb.2014.2331}.

Viana, M, R. Mancy, R. Biek, S. Cleaveland, P. C. Cross, J. O.
Lloyd-Smith, and D. T. Haydon (2014). ``Assembling evidence for
identifying reservoirs of infection''. In: \emph{Trends in Ecology \&
Evolution} 29.5, pp.~270-9. DOI:
\href{https://doi.org/10.1016\%2Fj.tree.2014.03.002}{10.1016/j.tree.2014.03.002}.

\hypertarget{section-6}{%
\subsection{2013}\label{section-6}}

Brennan, A, P. C. Cross, D. E. Ausband, A. Barbknecht, and S. Creel
(2013). ``Testing automated howling devices in a wintertime wolf
survey''. In: \emph{Wildlife Society Bulletin} 37.2, pp.~389-393. DOI:
\href{https://doi.org/10.1002\%2Fwsb.269}{10.1002/wsb.269}.

Brennan, A, P. C. Cross, M. Higgs, J. P. Beckman, R. W. Klaver, B.
Scurlock, and S. Creel (2013). ``Inferential consequences of modeling
rather than measuring snow accumulation in studies of animal ecology''.
In: \emph{Ecological Applications} 23.3, pp.~643-653. DOI:
\href{https://doi.org/10.1890\%2F12-0959.1}{10.1890/12-0959.1}.

Bright, P. R, H. T. Buxton, L. S. Balistrieri, L. B. Barber, F. H.
Chapelle, P. C. Cross, D. P. Krabbenhoft, G. S. Plumlee, J. M. Sleeman,
D. E. Tillitt, P. L. Toccalino, and J. R. Winton (2013). \emph{USGS
Environmental Health Science Strategy --- Providing Environmental Health
Science for a Changing World}. U.S. Geological Survey. URL:
\url{https://pubs.er.usgs.gov/publication/ofr20121069}.

Cassirer, E. F, R. K. Plowright, K. R. Manlove, P. C. Cross, A. P.
Dobson, K. A. Potter, P. J. Hudson, and A. White (2013).
``Spatio-temporal dynamics of pneumonia in bighorn sheep''. In:
\emph{Journal of Animal Ecology} 82, pp.~518-528. ISSN: 00218790. DOI:
\href{https://doi.org/10.1111\%2F1365-2656.12031}{10.1111/1365-2656.12031}.

Cross, P. C, D. Caillaud, and D. M. Heisey (2013). ``Underestimating the
effects of spatial heterogeneity due to individual movement and spatial
scale: infectious disease as an example''. In: \emph{Landscape Ecology}
28.2, pp.~247-257. DOI:
\href{https://doi.org/10.1007\%2Fs10980-012-9830-4}{10.1007/s10980-012-9830-4}.

Cross, P. C, T. Creech, M. Ebinger, K. Manlove, K. Irvine, J.
Henningsen, J. Rogerson, B. Scurlock, and S. Creel (2013). ``Female elk
contacts are neither frequency nor density dependent''. In:
\emph{Ecology} 94.9, pp.~2076-2086. DOI:
\href{https://doi.org/10.1890\%2F12-2086.1}{10.1890/12-2086.1}.

Cross, P. C, E. Maichak, A. Brennan, B. Scurlock, J. Henningsen, and G.
Luikart (2013). ``An ecological perspective on Brucella abortus in the
western United States''. In: \emph{Rev sci tech Off int Epiz} 32.1,
pp.~79-87.

Joseph, M. B, J. R. Mihaljevic, A. L. Arellano, J. G. Kueneman, D. L.
Preston, P. C. Cross, and P. T. J. Johnson (2013). ``Taming wildlife
disease: bridging the gap between science and management''. In:
\emph{Journal of Applied Ecology} 50, pp.~702-712. ISSN: 00218901. DOI:
\href{https://doi.org/10.1111\%2F1365-2664.12084}{10.1111/1365-2664.12084}.

Plowright, R, K. Manlove, E. F. Cassirer, P. C. Cross, T. Besser, and P.
Hudson (2013). ``Use of exposure history to identify patters of immunity
to pneumonia in bighorn sheep (Ovis canadensis)''. In: \emph{PLoS ONE}
8.4, p.~e61919. DOI:
\href{https://doi.org/10.1371\%2Fjournal.pone.0061919.g001}{10.1371/journal.pone.0061919.g001}.

Powell, J. H, S. T. Kalinowski, M. D. Higgs, M. R. Ebinger, N. V. Vu,
and P. C. Cross (2013). ``Microsatellites indicate minimal barriers to
mule deer Odocoileus hemionus dispersal across Montana , USA''. In:
\emph{Wildlife Biology} 19, pp.~102-110. DOI:
\href{https://doi.org/10.2981\%2F11-081}{10.2981/11-081}.

\hypertarget{section-7}{%
\subsection{2012}\label{section-7}}

Almberg, E. S, P. C. Cross, A. P. Dobson, D. W. Smith, and P. J. Hudson
(2012). ``Parasite invasion following host reintroduction: a case study
of Yellowstone's wolves''. In: \emph{Philosophical Transactions of the
Royal Society B: Biological Sciences} 367.1604, pp.~2840-2851. DOI:
\href{https://doi.org/10.1098\%2Frstb.2011.0369}{10.1098/rstb.2011.0369}.

Creech, T, P. C. Cross, B. Scurlock, E. Maichak, J. Rogerson, J.
Henningsen, and S. Creel (2012). ``Effects of low-density feeding on
elk-fetus contact rates on Wyoming feedgrounds''. In: \emph{Journal of
Wildlife Management} 76.5, pp.~877-886. DOI:
\href{https://doi.org/10.1002\%2Fjwmg.331}{10.1002/jwmg.331}.

Cross, P. C, T. G. Creech, M. R. Ebinger, D. M. Heisey, K. Irvine, and
S. Creel (2012). ``Wildlife contact analysis: emerging methods,
questions, and challenges''. In: \emph{Behavioral Ecology and
Sociobiolgy} 66.10, pp.~1437-1447. DOI:
\href{https://doi.org/10.1007\%2Fs00265-012-1376-6}{10.1007/s00265-012-1376-6}.

Forristal, V. E, S. Creel, M. Taper, B. Scurlock, and P. C. Cross
(2012). ``Effects of supplemental feeding and aggregation on fecal
glucocorticoid metabolite concentrations in elk''. In: \emph{Journal of
Wildlife Management} 76.4, pp.~694-702. DOI:
\href{https://doi.org/10.1002\%2Fjwmg.312}{10.1002/jwmg.312}.

Plowright, R. K, P. C. Cross, G. M. Tabor, E. S. Almberg, L. Bienen, and
P. J. Hudson (2012). ``Anthropogenic change and conservation medicine''.
In: \emph{New Directions in Conservation Medicine: Applied Cases of
Ecological Health}. Ed. by A. Aguirre, R. S. Ostfeld and P. Daszak. New
York: Oxford University Press. Chap. 8, pp.~111-121.

Ryan, S. J, P. C. Cross, J. Winnie, C. Hay, J. Bowers, and W. M. Getz
(2012). ``The utility of normalized difference vegetation index for
predicting African buffalo forage quality''. In: \emph{Journal of
Wildlife Management} 76.7, pp.~1499-1508. DOI:
\href{https://doi.org/10.1002\%2Fjwmg.407}{10.1002/jwmg.407}.

\hypertarget{section-8}{%
\subsection{2011}\label{section-8}}

Almberg, E. S, L. D. Mech, P. C. Cross, D. W. Smith, J. Sheldon, and R.
Crabtree (2011). ``Infectious disease in Yellowstone National Park's
canid community''. In: \emph{Yellowstone Science} 19.2, pp.~16-25. URL:
\url{http://pubs.er.usgs.gov/publication/70044063}.

Almberg, E, P. C. Cross, C. Johnson, D. Heisey, and B. Richards (2011).
``Modeling routes of chronic wasting disease transmission: Environmental
prion persistence promotes deer population decline and extinction''. In:
\emph{PLoS ONE} 6.5, p.~e19896. DOI:
\href{https://doi.org/10.1371\%2Fjournal.pone.0019896}{10.1371/journal.pone.0019896}.

Bai, Y, P. C. Cross, L. Malania, and M. Kosoy (2011). ``Isolation of
Bartonella capreoli from elk''. In: \emph{Veterinary Microbiology}
148.2-4, pp.~329-32. DOI:
\href{https://doi.org/10.1016\%2Fj.vetmic.2010.09.022}{10.1016/j.vetmic.2010.09.022}.

Ebinger, M. R, P. C. Cross, R. L. Wallen, P. J. White, and J. Treanor
(2011). ``Simulating sterilization , vaccination , and test-and-remove
as brucellosis control measures in bison''. In: \emph{Ecological
Applications} 21.8, pp.~2944-2959. DOI:
\href{https://doi.org/10.1890\%2F10-2239.1}{10.1890/10-2239.1}.

LaBeaud, A, P. Cross, W. Getz, and C. King (2011). ``Rift Valley Fever
Virus infection in African Buffalo (Syncerus caffer) herds in rural
South Africa: Evidence of interepidemic transmission''. In:
\emph{American Journal of Tropical Medicine and Hygiene} 89.4,
pp.~641-646. DOI:
\href{https://doi.org/10.4269\%2Fajtmh.2011.10-0187}{10.4269/ajtmh.2011.10-0187}.

Serrano, E, P. C. Cross, M. Beneria, A. Ficapal, J. Curia, X. Marco, S.
Lavin, and I. Marco (2011). ``Decreasing prevalence of brucellosis in
red deer through efforts to control disease in livestock''. In:
\emph{Epidemiology and infection} 139.10, pp.~1626-1630. DOI:
\href{https://doi.org/10.1017\%2FS0950268811000951}{10.1017/S0950268811000951}.

\hypertarget{section-9}{%
\subsection{2010}\label{section-9}}

Almberg, E, P. C. Cross, and D. Smith (2010). ``Persistence of canine
distemper virus in the Greater Yellowstone Ecosystem's carnivore
community''. In: \emph{Ecological Applications} 20.7, pp.~2058-2074.
DOI: \href{https://doi.org/10.1890\%2F09-1225.1}{10.1890/09-1225.1}.

Cross, P. C, E. Cole, A. Dobson, W. H. Edwards, K. L. Hamlin, G.
Luikart, A. Middleton, B. Scurlock, and P. White (2010). ``Probable
causes of increasing elk brucellosis in the Greater Yellowstone
Ecosystem''. In: \emph{Ecological Applications} 20.1, pp.~278-288. DOI:
\href{https://doi.org/10.1890\%2F08-2062.1}{10.1890/08-2062.1}.

Cross, P. C, M. Ebinger, V. Patrek, and R. Wallen (2010). ``Brucellosis
in cattle, bison, and elk: Management conflicts in a society with
diverse values''. In: \emph{Knowing Yellowstone: Science in America's
First National Park}. Ed. by J. Johnson. Boulder, CO: Taylor Trade
Publishing, pp.~81-94.

Cross, P. C, D. Heisey, B. Scurlock, W. H. Edwards, M. Ebinger, and A.
Brennan (2010). ``Mapping brucellosis increases relative to elk density
using hierarchical Bayesian models''. In: \emph{PLoS ONE} 5.4,
p.~e10322. DOI:
\href{https://doi.org/10.1371\%2Fjournal.pone.0010322}{10.1371/journal.pone.0010322}.

Heisey, D. M, E. E. Osnas, P. C. Cross, D. O. Joly, J. A. Langenberg,
and M. W. Miller (2010a). ``Linking process to pattern: estimating
spatiotemporal dynamics of a wildlife epidemic from cross-sectional
data''. In: \emph{Ecological Monographs} 80.2, pp.~221-240. ISSN:
0012-9615. DOI: \href{https://doi.org/Doi\%2010.1890\%2F09-0052.1}{Doi
10.1890/09-0052.1}.

Heisey, D, E. E. Osnas, P. C. Cross, D. Joly, J. A. Langenberg, and M.
Miller (2010b). ``Rejoiner: sifting through model space''. In:
\emph{Ecology} 91.12, pp.~3503-3514. DOI:
\href{https://doi.org/10.1890\%2F10-0894.1}{10.1890/10-0894.1}.

Polansky, L, G. Wittemyer, P. C. Cross, C. Tambling, and W. M. Getz
(2010). ``From moonlight to movement and synchronized randomness:
Fourier and wavelet analyses of animal location time series data''. In:
\emph{Ecology} 91.5, pp.~1506-1518. DOI:
\href{https://doi.org/10.1890\%2F08-2159.1}{10.1890/08-2159.1}.

Wittekindt, N, A. Padhi, S. Schuster, J. Qi, F. Zhao, L. Tomsho, L.
Kasson, M. Packard, P. C. Cross, and M. Poss (2010). ``Lymph node
meta-transcriptomics: exploring the host microbiome''. In: \emph{PLoS
ONE} 5.10, p.~e13432. DOI:
\href{https://doi.org/10.1371\%2Fjournal.pone.0013432}{10.1371/journal.pone.0013432}.

\hypertarget{section-10}{%
\subsection{2009}\label{section-10}}

Bar-David, S, I. Bar-David, P. C. Cross, S. Ryan, C. Knechtel, and W. M.
Getz (2009). ``Methods for assessing movement path recursion with
application ot African buffalo in South Africa''. In: \emph{Ecology}
90.9, pp.~2467-2479. DOI:
\href{https://doi.org/10.1890\%2F08-1532.1}{10.1890/08-1532.1}.

Cross, P. C, J. Drewe, V. Patrek, G. Pearce, M. D. Samuel, and R.
Delahay (2009). ``Wildlife population structure and parasite
transmission: Implications for disease management''. In:
\emph{Management of Disease in Wild Mammals}. Ed. by R. Delahay, G. C.
Smith and M. R. Hutchings. Tokyo: Springer. Chap. 2, pp.~9-30.

Cross, P. C, D. Heisey, J. A. Bowers, C. T. Hay, J. Wolhuter, P. Buss,
M. Hofmeyr, A. Michel, R. Bengis, T. Bird, I. Whyte, J. Du Toit, and W.
M. Getz (2009). ``Disease, predation and demography: assessing the
impacts of bovine tuberculosis on African buffalo by monitoring at
individual and population levels''. In: \emph{Journal of Applied
Ecology} 46, pp.~467-475. DOI:
\href{https://doi.org/10.1111\%2Fj.1365-2664.2008.01589.x}{10.1111/j.1365-2664.2008.01589.x}.

Maichak, E, B. Scurlock, J. Rogerson, L. Meadows, A. Barbknecht, W. H.
Edwards, and P. C. Cross (2009). ``Effects of management, behavior, and
scavenging on risk of brucellosis transmission in elk of western
Wyoming''. In: \emph{Journal of Wildlife Diseases} 45.2, pp.~398-410.
DOI:
\href{https://doi.org/10.7589\%2F0090-3558-45.2.398}{10.7589/0090-3558-45.2.398}.

Oosthuizen, W, P. Cross, J. Bowers, C. Hay, M. Ebinger, P. Buss, M.
Hofmeyr, and E. Z. Cameron (2009). ``Effects of chemical immobilization
on survival of African buffalo in the Kruger National Park''. In:
\emph{Journal of Wildlife Management} 73.1, pp.~149-153. DOI:
\href{https://doi.org/10.2193\%2F2008-071}{10.2193/2008-071}.

Wolhuter, J, R. Bengis, B. Reilly, and P. C. Cross (2009). ``Clinical
demodicosis in African Buffalo (Syncerus caffer) in the Kruger National
Park''. In: \emph{Journal of Wildlife Diseases} 45.2, pp.~502-504. DOI:
\href{https://doi.org/10.7589\%2F0090-3558-45.2.502}{10.7589/0090-3558-45.2.502}.

\hypertarget{section-11}{%
\subsection{2008}\label{section-11}}

Conner, M. C, M. Ebinger, J. A. Blanchong, and P. C. Cross (2008).
``Infectious disease in cervids of North America: Data, models, and
management challenges''. In: \emph{Annals of the New York Academy of
Sciences} 1134, pp.~146-172. DOI:
\href{https://doi.org/10.1196\%2Fannals.1439.005}{10.1196/annals.1439.005}.

Hay, C. T, P. C. Cross, and P. J. Funston (2008). ``Trade-offs between
predation and foraging explain sexual segregation in African buffalo''.
In: \emph{Journal of Animal Ecology} 77, pp.~850-858. DOI:
\href{https://doi.org/10.1111\%2Fj.1365-2656.2008.01409.x}{10.1111/j.1365-2656.2008.01409.x}.

Winnie, J. J, P. C. Cross, and W. M. Getz (2008). ``Habitat quality and
heterogeneity influence distribution and behavior in African Buffalo
(Syncerus caffer)''. In: \emph{Ecology} 89.5, pp.~1457-1468. DOI:
\href{https://doi.org/10.1890\%2F07-0772.1}{10.1890/07-0772.1}.

\hypertarget{section-12}{%
\subsection{2007}\label{section-12}}

Conner, M. M, J. Gross, P. C. Cross, M. D. Samuel, D. McKinnon, and M.
Miller (2007). \emph{Scale-dependent approaches to modeling spatial
epidemiology of chronic wasting disease}. Utah Division of Wildlife
Resources. URL: \url{http://wildlife.utah.gov/diseases/cwd/e-book/}.

Cross, P. C, W. H. Edwards, B. Scurlock, E. Maichak, and J. Rogerson
(2007). ``Effects of management and climate on elk brucellosis in the
Greater Yellowstone Ecosystem''. In: \emph{Ecological Applications}
17.4, pp.~957-964. DOI:
\href{https://doi.org/10.1890\%2F06-1603}{10.1890/06-1603}.

Cross, P. C, P. L. Johnson, J. O. Lloyd-Smith, and W. M. Getz (2007).
``Utility of R0 as a predictor of disease invasion in structured
populations''. In: \emph{Journal of the Royal Society Interface} 4,
pp.~315-324. DOI:
\href{https://doi.org/10.1098\%2Frsif.2006.0185}{10.1098/rsif.2006.0185}.

Cross, P. C. and G. Plumb (2007). ``Wildlife health initiatives in
Yellowstone National Park''. In: \emph{Yellowstone Science} 15.2,
pp.~4-7. URL:
\url{http://www.nps.gov/yell/planyourvisit/yellsci-issues.htm}.

Getz, W. M, S. Fortmann-Roe, P. Cross, A. Lyons, S. Ryan, and C. Wilmers
(2007). ``LoCoH: nonparameteric kernel methods for constructing home
ranges and utilization distributions''. In: \emph{PLoS ONE} 2.2,
p.~e207. DOI:
\href{https://doi.org/10.1371\%2Fjournal.pone.0000207}{10.1371/journal.pone.0000207}.

Hines, A, V. O. Ezenwa, P. C. Cross, and J. Rogerson (2007). ``Effects
of supplemental feeding on gastrointestinal parasite infection in elk
(Cervus elaphus): Preliminary observations''. In: \emph{Veterinary
Parasitology} 148.3-4, pp.~350-355. DOI:
\href{https://doi.org/10.1016\%2Fj.vetpar.2007.07.006}{10.1016/j.vetpar.2007.07.006}.

\hypertarget{section-13}{%
\subsection{2006}\label{section-13}}

Cross, P. C. and W. M. Getz (2006). ``Assessing vaccination as a control
strategy in an ongoing epidemic: Bovine tuberculosis in African
Buffalo''. In: \emph{Ecological Modelling} 196, pp.~494-504. DOI:
\href{https://doi.org/10.1016\%2Fj.ecolmodel.2006.02.009}{10.1016/j.ecolmodel.2006.02.009}.

Getz, W. M, J. O. Lloyd-Smith, P. C. Cross, S. Bar-David, P. L. Johnson,
T. C. Porco, and M. S. Sánchez (2006). ``Modeling the invasion and
spread of contagious disease in heterogeneous populations''. In:
\emph{Disease Evolution: Models, Concepts and Data Analyses}. Ed. by Z.
Feng, U. Dieckmann and S. A. Levin. Vol. 71. AMS-DIMACS, pp.~113-144.
ISBN: 978-1-4704-4028-2.

Michel, A. L, R. G. Bengis, D. F. Keet, M. Hofmeyr, L. M. de Klerk, P.
C. Cross, A. E. Jolles, D. Cooper, I. J. Whyte, P. Buss, and J. Godfroid
(2006). ``Wildlife tuberculosis in South African conservation areas:
Implications and challenges''. In: \emph{Veterinary Microbiology}
112.2-4, pp.~91-100. DOI:
\href{https://doi.org/\%3A10.1016\%2Fj.vetmic.2005.11.035}{:10.1016/j.vetmic.2005.11.035}.

\hypertarget{section-14}{%
\subsection{2005}\label{section-14}}

Cross, P. C, J. O. Lloyd-Smith, and W. Getz (2005). ``Disentangling
association patterns in fission-fusion societies using African buffalo
as an example''. In: \emph{Animal Behavior} 69.2, pp.~499-506. DOI:
\href{https://doi.org/10.1016\%2Fj.anbehav.2004.08.006}{10.1016/j.anbehav.2004.08.006}.

Cross, P. C, J. O. Lloyd-Smith, P. L. Johnson, and W. M. Getz (2005).
``Duelling timescales of host mixing and disease recovery determine
disease invasion in structured populations''. In: \emph{Ecology Letters}
8, pp.~587-595. DOI:
\href{https://doi.org/10.1111\%2Fj.1461-0248.2005.00760.x}{10.1111/j.1461-0248.2005.00760.x}.

Lloyd-Smith, J. O, P. C. Cross, C. J. Briggs, M. Daugherty, W. M. Getz,
J. Latto, M. S. Sanchez, A. B. Smith, and A. Swei (2005). ``Should we
expect population thresholds for wildlife disease?'' In: \emph{Trends in
Ecology \& Evolution} 20.9, pp.~511-519. DOI:
\href{https://doi.org/10.1016\%2Fj.tree.2005.07.004}{10.1016/j.tree.2005.07.004}.

\hypertarget{section-15}{%
\subsection{2004}\label{section-15}}

Cross, P. C, J. O. Lloyd-Smith, J. Bowers, C. Hay, M. Hofmeyr, and W. M.
Getz (2004). ``Integrating association data and disease dynamics in a
social ungulate: bovine tuberculosis in African buffalo in the Kruger
National Park''. In: \emph{Annales Zoologici Fennici} 41, pp.~879-892.
URL: \url{http://www.jstor.org/stable/23736148}.

Macandza, V, N. Owen-Smith, and P. C. Cross (2004). ``Forage selection
by African buffalo (Syncerus caffer) through the dry season in two
landscapes of the Kruger National Park''. In: \emph{South African
Journal of Wildlife Research} 34.2, pp.~113-121.

\hypertarget{section-16}{%
\subsection{2003}\label{section-16}}

Caron, A, P. C. Cross, and J. du Toit (2003). ``Ecological implications
of bovine tuberculosis in African Buffalo herds''. In: \emph{Ecological
Applications} 13.5, pp.~1338-1345. DOI:
\href{https://doi.org/10.1890\%2F02-5266}{10.1890/02-5266}.

\hypertarget{grey-literature}{%
\subsection{Grey Literature}\label{grey-literature}}

Rayl, ND, Proffitt, KM, Almberg, ES, Merkle, JA, Jones, JH, Gude, JA \&
Cross, PC. 2018 Modelling elk-to-livestock transmission risk to identify
hotspots of brucellosis spillover. pp.~1-56, Montana Fish, Wildlife and
Parks.

Ebinger, MR \& PC Cross. 2008. Surveillance for brucellosis in
Yellowstone bison: The power of various strategies to detect vaccination
effects. National Park Service, Mammoth, WY, YCR-2008-04. 69 pages.

\hypertarget{databases}{%
\section{Databases}\label{databases}}

van Manen, F.T., Smith, D.W., Haroldson, M.A., Stahler, D.R., Almberg,
E.S., Whitman, C.L., and Cross, P.C., 2018, Canine distemper virus
antibody titer results for grizzly bears and wolves in the Greater
Yellowstone Ecosystem 1984-2014: U.S. Geological Survey data release,
\url{https://doi.org/10.5066/P96E4UCK}.

Merkle, JA, PC Cross, BM Scurlock, EK Cole, AB Courtemanch, SR Dewey, MJ
Kauffman, and KE Szcodronski, 2017, Elk movement and predicted number of
brucellosis-induced abortion events in the southern Greater Yellowstone
Ecosystem (1993-2015): U.S. Geological Survey data release,
\url{https://doi.org/10.5066/F7474803}.

Brennan, A., Courtemanch, A.B., Cole, E.K., Dewey, S.R., and Cross,
P.C., 2018, Elk GPS collar data from National Elk Refuge (2006-2015):
U.S. Geological Survey data release,
\url{https://doi.org/10.5066/F7FF3RNW}.

Cross, PC, DM Heisey, JA Bowers, CT. Hay, J Wolhuter, P Buss, M Hofmeyr,
A Michel, R Bengis, T Bird, IJ Whyte, JT Du Toit, and WM Getz. 2009.
Buffalo herd tracking with VHF and GPS data.
\url{http://www.Movebank.org}.

Cross PC, Heisey DM, Bowers JA, Hay CT, Wolhuter J, Buss P, Hofmeyr M,
Michel AL, Bengis RG, Bird TLF, Du Toit JT, Getz WM (2008) Data from:
Disease, predation and demography: assessing the impacts of bovine
tuberculosis on African buffalo by monitoring at individual and
population levels. Dryad Digital Repository.
\url{http://dx.doi.org/10.5061/dryad.5hh3h}

Gorsich EE, Ezenwa VO, Cross PC, Bengis RG, Jolles AE (2015) Data from:
Context-dependent survival, fecundity, and predicted population-level
consequences of brucellosis in African buffalo. Dryad Digital Repository
\url{http://dx.doi.org/10.5061/dryad.p6678}.

\hypertarget{students}{%
\section{Students}\label{students}}

Gavin Cotterill. In progress. Managing disease in the supplemental
feeding grounds of Wyoming. Utah State University. Co-supervisor: JT Du
Toit\\
Ellen Brandell. In progress. Disease impacts on wolves in Yellowstone
National Park. Penn State University. Co-supervisor: PJ Hudson\\
Angela Brennan. 2014. Broad-scale determinants of elk aggregation and
brucellosis seroprevalence. Montana State University. Co-supervisor: S
Creel.\\
Tyler Creech. 2011. Heterogeneity in the fine-scale contact patterns of
elk as determined by proximity collars. Montana State University.
Co-supervisor: S Creel\\
Victoria Forristal (formerly Patrek). 2009. Masters. Fat but not happy:
The Effects of Supplemental Feeding on Stress Hormone Levels in Elk.
Montana State University. Co-supervisors: M Taper, S Creel\\
Craig Hay. 2006 Choice of Social environment of male buffalo
(\emph{Syncerus caffer}) in the Kruger National Park, South Africa.
Tshwane University of Technology. South Africa. Co-supervisor: P
Funston\\
Chris Oosthuizen. 2006. Honour's thesis: Chemical immobilization of
African buffalo (\emph{Syncerus caffer}) in Kruger National Park:
Evaluating effects on survival and reproduction. University of Pretoria.
South Africa. Co-supervisor: E Cameron

\hypertarget{post-docs}{%
\subsection{Post-docs}\label{post-docs}}

Alynn Martin. 2019-present. USGS. Brucellosis and bighorn sheep
disease.\\
Nathaniel Rayl. 2016-2018. USGS. Montana livestock risk assessments.\\
Kezia Manlove. 2016-2018. Washington State University. Bighorn sheep
pneumonia. Co-supervisor: Tom Besser. USDA funding.\\
Angela Brennan. 2015-2016. Montana State University. Livestock-wildlife
risk assessments. USDA funding.\\
Julio Benavides. 2014-2015. Montana State University. Epidemiology of
elk brucellosis. Co-supervisor: Scott Creel.\\
Aaron Foley. 2014-2015 USGS and Montana State University. Elk migration
and cow-calf ratios in western Wyoming. Co-supervisor: Scott Creel.
Partially funded by the U.S. Fish and Wildlife Service.\\
Pauline Kamath. 2011-2015. USGS. Population genetics and wildlife
disease.

\hypertarget{field-supervisor}{%
\subsection{Field Supervisor}\label{field-supervisor}}

Manlove, K. 2017. Penn State University. Supervisor: PJ Hudson\\
Almberg, ES. 2015. Penn State University. Supervisor: PJ Hudson\\
Bowers, JA. 2006. Master's thesis: Feeding patch selection of African
Buffalo (\emph{Syncerus caffer} in the central region of the Kruger
National Park.Tshwane University of Technology. South Africa\\
Tania Bird. 2004. Master's thesis: Influence of bovine tuberculosis
(\emph{Mycobacterium bovis}) on condition and reproductive success of
females African buffalo (\emph{Syncerus caffer}) in Kruger National
Park. University of Pretoria. South Africa\\
Shane Abeare. 2004. Master's thesis: Dry season habitat and patch
selection by African buffalo herds: test of a new home range estimator.
University of Pretoria. South Africa\\
Valerio Macandza 2002. Master's thesis: Forage selection by African
buffalo in the late dry season in two landscapes. Witwatersrand
University. South Africa\\
Alex Caron. 2001. Master's thesis: Ecological implications of bovine
tuberculosis in African Buffalo. University of Pretoria. South Africa

\hypertarget{teaching}{%
\section{Teaching}\label{teaching}}

Data Analysis and Multi-level / Hierarchical Modeling in Ecology (1
credit) Fall 2009. MSU.\\
Modeling Infectious Disease (3 day workshop) 2009. Univ. of Montana.\\
EcoLunch Seminar (1 credit) Fall 2008. MSU.\\
Plant-Disease Invasion Seminar (graduate seminar). Fall 2006. MSU.\\
Wildlife Ecology (4 credits with lab) Spring 2005. Co-Lecturer UC
Berkeley.\\
Disease Ecology (1 credit) 2004. Co-supervised graduate seminar on
disease ecology, UC Berkeley\\
Modeling Infectious Disease (1 week short-course) 2001. Univ. of
Witwatersrand, South Africa

\hypertarget{grantsawards}{%
\section{Grants/Awards}\label{grantsawards}}

USGS Performance Award (2018-2012,2010-2007)\\
USGS Cyclical \$30,000 Chronic wasting disease \hfill 2018\\
USGS Environmental Health Mission Area \$120,000 Chronic wasting
disease\\
NSF GRIP Supervisor: Fusobacterium and the elk microbiome\\
USDA grant to Univ of Washington, PI. \$96,000 Livestock-wildlife
disease modeling \hfill 2016\\
USGS Grade Promotion to GS-15\\
MT Fish Wildlife and Parks: \$90,000 Brucellosis risk assessment
\hfill 2015\\
USGS \$133,000 Greater Yellowstone ecosystem disease research.\\
NIMBioS Workshop (Co-PI) \textasciitilde{}\$60,000 \hfill 2013\\
NSF Dissertation Improvement Grant (Co-PI) \$19,343\\
USGS, PI \$ 98,000 Greater Yellowstone ecosystem disease research\\
USGS Powell Center Grant, Co-PI (declined) \hfill 2012\\
USGS, PI \$75,000, Disease effects on Yellowstone Wolves.\\
USFWS, PI\$45,000, Elk space-use of the National Elk Refuge.\\
Morris Animal Foundation, Co-PI \$ 75,000.\\
USGS Best Paper in Biology \hfill 2011\\
USGS Grade Promotion\\
NSF-NIH Ecology of Infectious Disease Program, co-PI \$1,971,033,
\hfill 2010\\
USGS, PI \$75,000, Park Oriented Biological Support Grant\\
USGS, PI \$39,000, Modeling environmental transmission of Chronic
Wasting Disease\hfill 2009\\
USGS, Co-PI \$320,000, Global Climate Change Initiative \hfill 2008\\
Co-PI \$112,180, Wyoming Livestock-Wildlife Disease Initiative\\
Co-PI \$281,000, Wyoming Game and Fish Department: Tracking elk
movements.\\
USGS, Co-PI \$750,000 with Mary Poss (Penn State): Viral tracking of
mule deer and elk. \hfill 2007\\
NPS, PI \$10,000: Brucellosis in Yellowstone National Park \hfill 2006\\
USGS, PI \$210,000 for chronic wasting disease research. \hfill 2005\\
NSF-NIH Ecology of Infectious Disease Grant. \$1.8 million. Initiated,
co-authored, and developed the research program with Dr.~Wayne
Getz.\hfill 1999

\hypertarget{invited-presentations}{%
\section{Invited Presentations}\label{invited-presentations}}

Greater Yellowstone Coordinating Committee. CWD review. \hfill 2019\\
MSU Institute on Ecosystems Seminar. \hfill 2018\\
SERDP workshop on emerging infectious disease. Emigrant, MT.\\
Brucellosis review for state wildlife and livestock agency partners.
Bozeman, MT.\\
Plenary: International Deer Congress, Estes Park, CO.\\
Invited Panelist. Linking science and management. GYE Biennial Meeting,
Big Sky MT.\\
Virginia Tech workshop on mange.\\
Univ. of Minnesota Veterinary Medicine Seminar.\\
Glasgow University Ecology Seminar, Glasgow, UK. \hfill 2017\\
Chile, Universidad Andres Bello, Santiago.\\
University of California at Berkeley, Wildlife Seminar\\
Plenary, 12th Western States and Provinces Deer and Elk Workshop. Sun
Valley, ID.\\
Penn State University, Center for Infectious Disease Dynamics Seminar.\\
USDA/APHIS Briefing on the National Academy of Sciences Report.\\
Utah State University Ecology Seminar. Logan, UT \hfill 2016\\
UCLA Ecology Seminar, Los Angeles, CA.\\
Patuxent USGS Seminar.\\
Georgetown Ecology Seminar, Washington, D.C.\\
South Africa, 50th Anniv. Mammal Research Institute, South Africa\\
K-5 science and technology night, Bozeman MT.\\
MT Conservation Biology Evening Lecture, Bozeman MT \hfill 2015\\
The National Academy of Sciences, Washington DC\\
Wildlife Society Meeting, Winnepeg, Canada \hfill 2014\\
University of Sherbrooke Ecology Seminar, Canada\\
NIH Rocky Mountain Lab, Hamilton MT\\
Chile, 18th Congreso Chileno de Medicina Veterinaria\\
Chile, Univ. Catolica, Dept. Seminar\\
Public Talk, Emerging wildlife pandemics, Menlo Park, CA\\
Steering Committee \& Speaker. Foreign Animal Disease, Shepardstown WV
\hfill 2013\\
Interagency Bison Management Plan Meeting. Chico MT\\
European Conservation Biology Meeting, Glasgow UK \hfill 2012\\
Wildlife disease management workshop, Penn State Univ. \hfill 2011\\
Ecology and Evolution of Infectious Disease Meeting, Santa Barbara.\\
RAPPID-NIH Movement and Mosquito-Transmitted Diseases Meeting,
Washington D.C.\\
Keynote: Berryman Institute Biennial Meeting, Logan UT. \hfill 2010\\
RAPPID-NIH Movement and Mosquito-Transmitted Diseases Meeting,
Washington D.C.\\
Kopriva Lecture: MSU College of Arts and Science, Bozeman, MT
\hfill 2009\\
Colorado State University, Dept. Seminar, Fort Collins, CO\\
10th Biennial Conference of Research on the Colorado Plateau Speaker,
ESA,

\hypertarget{service}{%
\section{Service}\label{service}}

Associate Editor, Ecological Applications \hfill ongoing\\
Chair, USGS Animal Use and Care Committee for NOROCK.\\
Northern Yellowstone Cooperative Wildlife Working Group.\\
External Review of USDA Cattle Fever Tick Eradication Program.
\hfill 2018-2019\\
Associate Editor, Journal of Wildlife Management \hfill 2015-2017\\
National Academy of Science Panel Member: Revisiting Brucellosis in the
GYE\\
Dept. Homeland Security IPT for outbreak response and assessment tools.
\hfill 2016\\
Red Wolf Recovery Implementation Team, USFWS. \hfill 2014\\
Participant, Wildlife Conservation Society Wildlife Health Program,
Internal Strategic Workshop for future research.\\
Steering Committee, Group Leader \& Speaker. Foreign Animal Disease
National Preparedness Workshop. USGS/DHS/USDA/CDC. \hfill 2013\\
Member Environmental Health Strategic Science Planning Team
\hfill 2011\\
Co-coordinator \& originator, NIH \& DHS RAPIDD Working group on
cross-species transmission. 15 participants\\
Participant, NIH \& DHS RAPIDD Working group on movement and
mosquito-borne diseases.\\
USGS representative. Northern Rockies NEON committee.\\
Organizer, Greater Yellowstone Brucellosis Research Meeting (2 days), 60
participants, 27 speakers, Bozeman MT \hfill 2009\\
Participant, Dept. of Interior Avian Influenza Preparedness workshop,
Madison WI\\
Participant, Yellowstone National Park Science Agenda Workshop, Bozeman
MT\\
Steering Committee, Yellowstone National Park Wildlife Health Program
Meeting. \hfill 2007\\
Participant, USDA workshop: The Science of Surveillance, Control and
Eradication of Catastrophic Diseases in Wildlife, Pinagree Park CO\\
Participant, Disease and conservation of mammals, Conservation
International \hfill 2006\\
USGS representative. Greater Yellowstone Interagency Brucellosis
Committee \hfill 2005-8

\hypertarget{reviewer}{%
\section{Reviewer}\label{reviewer}}

\emph{Journals} (since 2004): Nature, Ecology, Ecol App, J Anim Ecol, J
App Ecol, Proc Roy Soc B, Phil Trans Roy Soc, Biol Letters, Cons Bio,
Biol Cons, Anim Cons, Biodiv Cons, Behavior, Envi Cons, J Wildl Dis,
Wildl Bio, PloS ONE, EcoHealth, J Theo Bio, SA J Wildl Res, Ann Zoo
Fennici, Ecol Mod

\emph{Funding Agencies}: National Science Foundation, Wildlife
Conservation Society, Wellcome Trust, Biotechnology and Biological
Sciences Research Council UK, Natural Environment Research Council UK,
South African National Research Foundation, Alberta Prion Research
Institute, National Institutes of Health

\hypertarget{press-and-outreach}{%
\section{Press and Outreach}\label{press-and-outreach}}

\href{https://naturallyspeaking.blog/2017/04/26/episode-51-natures-greatest-theatre-ecology-and-disease-in-yellowstone/}{Naturally
Speaking} Podcast \hfill 2017\\
\href{http://www.bozemandailychronicle.com/news/environment/report-elk-greater-brucellosis-transmission-risk-than-bison/article_8329c551-18a2-50a5-9352-f585935a7d99.html}{Bozeman
Daily Chronicle}. National Academy of Sciences Brucellosis Report.
\hfill 2016\\
\href{https://missoulian.com/news/state-and-regional/disease-s-spread-blamed-on-elk-not-bison-or-feed/article_379e198f-190b-562a-bd6d-950fbed46f71.html}{Missoulian}
\href{https://billingsgazette.com/lifestyles/recreation/mange-changes-yellowstone-wolves-hunting-travel-and-food-needs/article_f876b43f-3e46-5d78-b6e5-ed5e8ddddbdb.html}{Billings
Gazette} Mange in Yellowstone\\
\href{http://www.economist.com/news/science-and-technology/21652259-wolves-yellowstone-provide-some-surprising-survival-lessons-pack-power}{The
Economist} Yellowstone Mange \hfill 2015\\
\href{https://vimeo.com/104296498}{Montana PBS} Silencing the Thunder
Brucellosis Documentary \hfill 2014\\
\href{https://science360.gov/obj/video/0f50aca7-2691-4126-996f-8ec5b74a9eb0/understanding-ecological-role-wolves-yellowstone-national-park}{NSF
Science 360} Yellowstone Mange.\\
\href{http://store.discoveryeducation.com/product/show/129481}{Discovery
Channel}: Curiosity X-Ray Yellowstone \hfill 2012\\
\href{https://www.wired.com/2012/05/st_photo_wolves/}{Wired Magazine}
Thermal imagery of Yellowstone Wolves\\
\href{http://scienceworld.scholastic.com/issues/09_17_12}{Science World
Scholastic Magazine} K-12 Education on Yellowstone Wolves\\
\href{http://www.yellowstonewolf.org}{Yellowstone Wolf Citizen Science
Webpage}

\hypertarget{references}{%
\section{References}\label{references}}

Dr.~Claudia Regan: Center Director and Direct Supervisor.\\
Northern Rocky Mountain Science Center, USGS, 2327 University Way, Suite
2, Bozeman MT 59715.\\
Phone: (406) 994-7972 Email:
\href{mailto:cregan@usgs.gov}{\nolinkurl{cregan@usgs.gov}}

Mr.~Brandon Scurlock: Collaborator.\\
Wildlife Management Coordinator, Wyoming Game and Fish Department,
Pinedale Office, PO Box 850, Pinedale, WY 82941.\\
Phone: (307) 367-4353 Email:
\href{mailto:bscurlock@wyo.gov}{\nolinkurl{bscurlock@wyo.gov}}

Dr.~Doug W. Smith: Collaborator.\\
Yellowstone Center for Resources, Wolf Project, Yellowstone National
Park, WY 82190\\
Phone: (307) 344-2242 Email:
\href{mailto:doug_smith@nps.gov}{\nolinkurl{doug\_smith@nps.gov}}

Dr.~Johan du Toit, Collaborator and previous advisor, Utah State
University, NR 206 Phone: (435) 797-2837 Email:
\href{mailto:johan.dutoit@usu.edu}{\nolinkurl{johan.dutoit@usu.edu}}

For the Rocky Crate Chair faculty position, the two references below are
also appropriate:

Dr.~Terry McElwain: Collaborator on National Academy Panel.\\
Regents Professor Emeritus, Washington State University, Pullman, WA
99164-7090\\
Email: \href{mailto:tfm@wsu.edu}{\nolinkurl{tfm@wsu.edu}}


\end{document}
